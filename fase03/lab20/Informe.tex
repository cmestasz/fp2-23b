%package list
\documentclass{article}
\usepackage[top=3cm, bottom=3cm, outer=3cm, inner=3cm]{geometry}
\usepackage{multicol}
\usepackage{graphicx}
\usepackage{url}
%\usepackage{cite}
\usepackage{hyperref}
\usepackage{array}
%\usepackage{multicol}
\newcolumntype{x}[1]{>{\centering\arraybackslash\hspace{0pt}}p{#1}}
\usepackage{natbib}
\usepackage{pdfpages}
\usepackage{multirow}
\usepackage[normalem]{ulem}
\useunder{\uline}{\ul}{}
\usepackage{svg}
\usepackage{xcolor}
\usepackage{listings}
\lstdefinestyle{ascii-tree}{
    literate={├}{|}1 {─}{--}1 {└}{+}1 
  }
\lstset{basicstyle=\ttfamily,
  showstringspaces=false,
  commentstyle=\color{red},
  keywordstyle=\color{blue}
}
%\usepackage{booktabs}
\usepackage[labelformat=empty]{caption}
\usepackage{subcaption}
\usepackage{float}
\usepackage{array}

\newcolumntype{M}[1]{>{\centering\arraybackslash}m{#1}}
\newcolumntype{N}{@{}m{0pt}@{}}


%%%%%%%%%%%%%%%%%%%%%%%%%%%%%%%%%%%%%%%%%%%%%%%%%%%%%%%%%%%%%%%%%%%%%%%%%%%%
%%%%%%%%%%%%%%%%%%%%%%%%%%%%%%%%%%%%%%%%%%%%%%%%%%%%%%%%%%%%%%%%%%%%%%%%%%%%
\newcommand{\itemEmail}{cmestasz@unsa.edu.pe}
\newcommand{\itemStudent}{Christian Mestas Zegarra}
\newcommand{\itemCourse}{Fundamentos de la Programación 2}
\newcommand{\itemCourseCode}{1701213}
\newcommand{\itemSemester}{II}
\newcommand{\itemUniversity}{Universidad Nacional de San Agustín de Arequipa}
\newcommand{\itemFaculty}{Facultad de Ingeniería de Producción y Servicios}
\newcommand{\itemDepartment}{Departamento Académico de Ingeniería de Sistemas e Informática}
\newcommand{\itemSchool}{Escuela Profesional de Ingeniería de Sistemas}
\newcommand{\itemAcademic}{2023 - B}
\newcommand{\itemInput}{Del 08 Enero 2024}
\newcommand{\itemOutput}{Al 15 Enero 2024}
\newcommand{\itemPracticeNumber}{20}
\newcommand{\itemTheme}{Clase Ejército - Soldado - Mapa. Herencia y Polimorfismo. Miembros de clase}
%%%%%%%%%%%%%%%%%%%%%%%%%%%%%%%%%%%%%%%%%%%%%%%%%%%%%%%%%%%%%%%%%%%%%%%%%%%%
%%%%%%%%%%%%%%%%%%%%%%%%%%%%%%%%%%%%%%%%%%%%%%%%%%%%%%%%%%%%%%%%%%%%%%%%%%%%

\usepackage[english,spanish]{babel}
\usepackage[utf8]{inputenc}
\AtBeginDocument{\selectlanguage{spanish}}
\renewcommand{\figurename}{Figura}
\renewcommand{\refname}{Referencias}
\renewcommand{\tablename}{Tabla} %esto no funciona cuando se usa babel
\AtBeginDocument{%
	\renewcommand\tablename{Tabla}
}

\usepackage{fancyhdr}
\pagestyle{fancy}
\fancyhf{}
\setlength{\headheight}{30pt}
\renewcommand{\headrulewidth}{1pt}
\renewcommand{\footrulewidth}{1pt}
\fancyhead[L]{\raisebox{-0.2\height}{\includegraphics[width=3cm]{img/logo_episunsa.png}}}
\fancyhead[C]{\fontsize{7}{7}\selectfont	\itemUniversity \\ \itemFaculty \\ \itemDepartment \\ \itemSchool \\ \textbf{\itemCourse}}
\fancyhead[R]{\raisebox{-0.2\height}{\includegraphics[width=1.2cm]{img/logo_abet}}}
\fancyfoot[L]{Christian Mestas}
\fancyfoot[C]{\itemCourse}
\fancyfoot[R]{Página \thepage}

% para el codigo fuente
\usepackage{listings}
\usepackage{color, colortbl}
\definecolor{dkgreen}{rgb}{0,0.6,0}
\definecolor{gray}{rgb}{0.5,0.5,0.5}
\definecolor{mauve}{rgb}{0.58,0,0.82}
\definecolor{codebackground}{rgb}{0.95, 0.95, 0.92}
\definecolor{tablebackground}{rgb}{0.8, 0, 0}

\lstset{frame=tb,
	language=bash,
	aboveskip=3mm,
	belowskip=3mm,
	showstringspaces=false,
	columns=flexible,
	basicstyle={\small\ttfamily},
	numbers=none,
	numberstyle=\tiny\color{gray},
	keywordstyle=\color{blue},
	commentstyle=\color{dkgreen},
	stringstyle=\color{mauve},
	breaklines=true,
	breakatwhitespace=true,
	tabsize=3,
	backgroundcolor= \color{codebackground},
}

\begin{document}

\vspace*{10px}

\begin{center}
	\fontsize{17}{17} \textbf{ Informe de Laboratorio \itemPracticeNumber}
\end{center}
\centerline{\textbf{\Large Tema: \itemTheme}}
%\vspace*{0.5cm}	

\begin{flushright}
	\begin{tabular}{|M{2.5cm}|N|}
		\hline
		\rowcolor{tablebackground}
		\color{white} \textbf{Nota} \\
		\hline
		\\[30pt]
		\hline
	\end{tabular}
\end{flushright}

\begin{table}[H]
	\begin{tabular}{|M{4.7cm}|M{4.8cm}|M{4.8cm}|}
		\hline
		\rowcolor{tablebackground}
		\color{white} \textbf{Estudiante} & \color{white}\textbf{Escuela} & \color{white}\textbf{Asignatura}                                        \\
		\hline
		{\itemStudent \par \itemEmail}    & \itemSchool                   & {\itemCourse \par Semestre: \itemSemester \par Código: \itemCourseCode} \\
		\hline
	\end{tabular}
\end{table}

\begin{table}[H]
	\begin{tabular}{|M{4.7cm}|M{4.8cm}|M{4.8cm}|}
		\hline
		\rowcolor{tablebackground}
		\color{white}\textbf{Laboratorio} & \color{white}\textbf{Tema} & \color{white}\textbf{Duración} \\
		\hline
		\itemPracticeNumber               & \itemTheme                 & 04 horas                       \\
		\hline
	\end{tabular}
\end{table}

\begin{table}[H]
	\begin{tabular}{|M{4.7cm}|M{4.8cm}|M{4.8cm}|}
		\hline
		\rowcolor{tablebackground}
		\color{white}\textbf{Semestre académico} & \color{white}\textbf{Fecha de inicio} & \color{white}\textbf{Fecha de entrega} \\
		\hline
		\itemAcademic                            & \itemInput                            & \itemOutput                            \\
		\hline
	\end{tabular}
\end{table}

%%%%%%%%%%%%%%%%%%%%%%%%%%%%%%%%%%%%%%%%%%%%%%%%%%%%%%%%%%%%%%%%%%%%%%
\section{Tarea}
\begin{itemize}
	\item \textbf{Item 1:} 
		  \\• Crear diagrama de clases UML y programa.
	      \\• Crear los miembros de cada clase de la forma más adecuada: como miembros
	      de clase o de instancia.
	      \\• Crear la clase Mapa, que esté constituida por el tablero antes visto, que
	      posicione soldados en ciertas posiciones aleatorias (entre 1 y 10 soldados por
	      cada ejército, sólo 1 ejército por reino). Se deben generar ejércitos de 2 reinos.
	      No se admite guerra civil. El Mapa tiene como atributo el tipo de territorio que
	      es (bosque, campo abierto, montaña, desierto, playa). La cantidad de soldados,
	      así como todos sus atributos se deben generar aleatoriamente.
	      \\• Dibujar el Mapa con las restricciones que sólo 1 soldado como máximo en cada
	      cuadrado.
	      \\• El mapa tiene un solo tipo de territorio.
	      \\• Considerar que el territorio influye en los resultados de las batallas, así cada
	      reino tiene bonus según el territorio: Inglaterra->bosque, Francia->campo
	      abierto, Castilla-Aragón->montaña, Moros->desierto, Sacro Imperio Romano-
	      Germánico->bosque, playa, campo abierto. En dichos casos, se aumenta el
	      nivel de vida en 1 a todos los soldados del reino beneficiado.
	      \\• En la historia, los ejércitos estaban conformados por diferentes tipos de
	      soldados, que tenían similitudes, pero también particularidades.
	      \\• Basándose en la clase Soldado crear las clases Espadachín, Arquero, Caballero
	      y Lancero. Las cuatro clases heredan de la superclase Soldado pero aumentan
	      atributos y métodos, o sobrescriben métodos heredados.
	      \\• Los espadachines tienen como atributo particular "longitud de espada" y como
	      acción "crear un muro de escudos" que es un tipo de defensa en particular.
	      \\• Los caballeros pueden alternar sus armas entre espada y lanza, además de
	      desmontar (sólo se realiza cuando está montando e implica defender y cambiar
	      de arma a espada), montar (sólo se realiza cuando está desmontado e implica
	      montar, cambiar de arma a lanza y envestir). El caballero también puede
	      envestir, ya sea montando o desmontando, cuando es desmontado equivale a
	      atacar 2 veces pero cuando está montando implica a atacar 3 veces.
	      \\• Los arqueros tienen un número de flechas disponibles las cuales pueden
	      dispararse y se gastan cuando se hace eso.
	      \\• Los lanceros tienen como atributo particular, "longitud de lanza" y como acción
	      "schiltrom" (como una falange que es un tipo de defensa en particular y que
	      aumenta su nivel de defensa en 1).
	      \\• Tendrá 2 Ejércitos que pueden ser constituidos sólo por espadachines,
	      caballeros, arqueros y lanceros. No se acepta guerra civil. Crear una estructura
	      de datos conveniente para el tablero. Los soldados del primer ejército se
	      almacenarán en un arreglo estándar y los soldados del segundo ejército se
	      almacenarán en un ArrayList. Cada soldado tendrá un nombre autogenerado:
	      Espadachin0X1, Arquero1X1, Caballero2X2, etc., un valor de nivel de vida
	      autogenerado aleatoriamente, la fila y columna también autogenerados
	      aleatoriamente (no puede haber 2 soldados en el mismo cuadrado) y valores
	      autogenerados para el resto de atributos.
	      \\• Todos los caballeros tendrán los siguientes valores: ataque 13, defensa 7, nivel
	      de vida [10..12] (el nivel de vida actual empieza con el valor del nivel de vida).
	      \\• Todos los arqueros tendrán los siguientes valores: ataque 7, defensa 3, nivel
	      de vida [3..5] (el nivel de vida actual empieza con el valor del nivel de vida).
	      \\• Todos los espadachines tendrán los siguientes valores: ataque 10, defensa 8,
	      nivel de vida [8..10] (el nivel de vida actual empieza con el valor del nivel de
	      vida).
	      \\• Todos los lanceros tendrán los siguientes valores: ataque 5, defensa 10, nivel
	      de vida [5..8] (el nivel de vida actual empieza con el valor del nivel de vida).
	      \\• Mostrar el tablero, distinguiendo los ejércitos y los tipos de soldados creados.
	      Además, se debe mostrar todos los datos de todos los soldados creados para
	      ambos ejércitos. Además de los datos del soldado con mayor vida de cada
	      ejército, el promedio de nivel de vida de todos los soldados creados por ejército,
	      los datos de todos los soldados por ejército en el orden que fueron creados y
	      un ranking de poder de todos los soldados creados por ejército (del que tiene
	      más nivel de vida al que tiene menos) usando algún algoritmo de ordenamiento.
	      \\• Finalmente, que muestre el resumen los 2 ejércitos, indicando el reino, cantidad
	      de unidades, distribución del ejército según las unidades, nivel de vida total del
	      ejército y qué ejército ganó la batalla (usar la métrica de suma de niveles de
	      vida y porcentajes de probabilidad de victoria basado en ella). Este porcentaje
	      también debe mostrarse.
	      \\• Hacerlo programa iterativo.
\end{itemize}
%%%%%%%%%%%%%%%%%%%%%%%%%%%%%%%%%%%%%%%%%%%%%%%%%%%%%%%%%%%%%%%%%%%%%%
\pagebreak

\section{Equipos, materiales y temas utilizados}
\begin{itemize}
	\item Sistema Operativo Microsoft Windows 10 Pro 64 bits
	\item Visual Studio Code 1.82.2
	\item Java Development Kit 17.0.1
	\item Git 2.41.0.windows.1
	\item Windows PowerShell 5.1.19041.3031
	\item Cuenta en GitHub con el correo institucional.
	      %%%%%%%%%%%%%%%%%%%%%%%%%%%%%%%%%%%%%%%%%%%%%%%%%%%%%%%%%%%%%%%%%%%%%%
	\item Programación Orientada a Objetos
	\item HashMap de Objetos
	\item ArrayList de Objetos
	\item Agregación y composición
	\item Herencia y polimorfismo
	\item Miembros de clase e instancia
	      %%%%%%%%%%%%%%%%%%%%%%%%%%%%%%%%%%%%%%%%%%%%%%%%%%%%%%%%%%%%%%%%%%%%%%
\end{itemize}

\section{URL de Repositorio Github}
\begin{itemize}
	\item URL del Repositorio GitHub para clonar o recuperar.
	\item \url{https://github.com/cmestasz/fp2-23b.git}
	      %%%%%%%%%%%%%%%%%%%%%%%%%%%%%%%%%%%%%%%%%%%%%%%%%%%%%%%%%%%%%%%%%%%%%%
	\item URL para el laboratorio 20 en el Repositorio GitHub.
	\item \url{https://github.com/cmestasz/fp2-23b/tree/main/fase03/lab20}
	      %%%%%%%%%%%%%%%%%%%%%%%%%%%%%%%%%%%%%%%%%%%%%%%%%%%%%%%%%%%%%%%%%%%%%%
\end{itemize}
\pagebreak

\section{Actividades con el repositorio GitHub}
\lstinputlisting[keepspaces=true,language=bash,caption={commits.bash},numbers=left,]{commits.bash}
\begin{figure}[H]
	\centering
	\includegraphics[width=1\textwidth,keepaspectratio]{img/commit01.jpg}
	\caption{Primer Commit.}
\end{figure}
\begin{figure}[H]
	\centering
	\includegraphics[width=1\textwidth,keepaspectratio]{img/commit02.jpg}
	\caption{Segundo Commit.}
\end{figure}
\begin{figure}[H]
	\centering
	\includegraphics[width=1\textwidth,keepaspectratio]{img/commit03.jpg}
	\caption{Tercer Commit.}
\end{figure}
\begin{figure}[H]
	\centering
	\includegraphics[width=1\textwidth,keepaspectratio]{img/commit04.jpg}
	\caption{Cuarto Commit.}
\end{figure}
\begin{figure}[H]
	\centering
	\includegraphics[width=1\textwidth,keepaspectratio]{img/commit05.jpg}
	\caption{Quinto Commit.}
\end{figure}
\begin{figure}[H]
	\centering
	\includegraphics[width=1\textwidth,keepaspectratio]{img/commit06.jpg}
	\caption{Sexto Commit.}
\end{figure}
\begin{figure}[H]
	\centering
	\includegraphics[width=1\textwidth,keepaspectratio]{img/commit07.jpg}
	\caption{Septimo Commit.}
\end{figure}
\begin{figure}[H]
	\centering
	\includegraphics[width=1\textwidth,keepaspectratio]{img/commit08.jpg}
	\caption{Octavo Commit.}
\end{figure}
\begin{figure}[H]
	\centering
	\includegraphics[width=1\textwidth,keepaspectratio]{img/commit09.jpg}
	\caption{Noveno Commit.}
\end{figure}
\begin{figure}[H]
	\centering
	\includegraphics[width=1\textwidth,keepaspectratio]{img/commit10.jpg}
	\caption{Decimo Commit.}
\end{figure}
\begin{figure}[H]
	\centering
	\includegraphics[width=1\textwidth,keepaspectratio]{img/commit11.jpg}
	\caption{Decimo Primer Commit.}
\end{figure}
\begin{figure}[H]
	\centering
	\includegraphics[width=1\textwidth,keepaspectratio]{img/commit12.jpg}
	\caption{Decimo Segundo Commit.}
\end{figure}
\pagebreak

\section{Código desarrollado}
\lstinputlisting[language=Java, caption={Soldado.java},numbers=left,]{Soldado.java}
\begin{itemize}
	\item Clase abstracta que guarda nombre y vida del soldado.
	\item Posee getters para todos los atributos.
	\item Posee métodos para el combate como aumentarVida, atacar, herir, defender.
\end{itemize}
\lstinputlisting[language=Java, caption={Caballero.java},numbers=left,]{Caballero.java}
\begin{itemize}
	\item Clase que mantiene atributos y métodos independientes de un caballero.
	\item Arma, montado, cambiarArma, montar, desmontar, embestir.
\end{itemize}
\lstinputlisting[language=Java, caption={Arquero.java},numbers=left,]{Arquero.java}
\begin{itemize}
	\item Clase que mantiene atributos y métodos independientes de un arquero.
	\item Flechas, disparar.
\end{itemize}
\lstinputlisting[language=Java, caption={Espadachin.java},numbers=left,]{Espadachin.java}
\begin{itemize}
	\item Clase que mantiene atributos y métodos independientes de un espadachin.
	\item Longitud de espada, generarMuroEscudos.
\end{itemize}
\lstinputlisting[language=Java, caption={Lancero.java},numbers=left,]{Lancero.java}
\begin{itemize}
	\item Clase que mantiene atributos y métodos independientes de un lancero.
	\item Longitud de lanza, schiltrom.
\end{itemize}
\lstinputlisting[language=Java, caption={Mapa.java},numbers=left,]{Mapa.java}
\begin{itemize}
	\item Ciclo de un juego contenido dentro del constructor.
	\item Método inicializarSoldados crea los soldados y los pone en el tablero.
	\item Método imprimirTablero imprime el tablero de juego.
	\item Método verificarVentaja otorga ventaja al reino aventajado por el terreno.
	\item Método imprimirEstado imprime el estado del tablero.
	\item Método imprimirGanador calcula el ganador de la batalla y lo imprime.
\end{itemize}
\lstinputlisting[language=Java, caption={Videojuego.java},numbers=left,]{Videojuego.java}
\begin{itemize}
	\item Permite el comportamiento iterativo creando nuevos mapas cada iteración.
\end{itemize}
\pagebreak

\section{Ejecución del código}
\lstinputlisting[keepspaces=true,language=bash,caption={VideoJuego9.java},numbers=left,]{ejec01.bash}
\pagebreak

\section{Diagrama UML}
\begin{figure}[H]
	\centering
	\includegraphics[width=1\textwidth,keepaspectratio]{img/uml.jpg}
	\caption{Diagrama UML.}
\end{figure}
\pagebreak

\section{Estructura de laboratorio \itemPracticeNumber}
\begin{itemize}
	\item El contenido que se entrega en este laboratorio es el siguiente:
\end{itemize}
%%%%%%%%%%%%%%%%%%%%%%%%%%%%%%%%%%%%%%%%%%%%%%%%%%%%%%%%%%%%%%%%%%%%%%
\begin{lstlisting}[style=ascii-tree]
lab20/
|--- Soldado.java
|--- Caballero.java
|--- Arquero.java
|--- Espadachin.java
|--- Lancero.java
|--- Mapa.java
|--- Videojuego.java
|--- commits.bash
|--- ejec01.bash
|--- Informe.tex
|--- Informe.pdf
|--- img
	|--- logo_abet.png
	|--- logo_episunsa.png
	|--- logo_unsa.jpg
	|--- commit01.jpg
	|--- commit02.jpg
	|--- commit03.jpg
	|--- commit04.jpg
	|--- commit05.jpg
	|--- commit06.jpg
	|--- commit07.jpg
	|--- commit08.jpg
	|--- commit09.jpg
	|--- commit10.jpg
	|--- commit11.jpg
	|--- commit12.jpg
	|--- uml.jpg
\end{lstlisting}
%%%%%%%%%%%%%%%%%%%%%%%%%%%%%%%%%%%%%%%%%%%%%%%%%%%%%%%%%%%%%%%%%%%%%%
\pagebreak

\section{\textcolor{red}{Rúbricas}}

\subsection{\textcolor{red}{Entregable Informe}}
\begin{table}[H]
	\caption{Tipo de Informe}
	\setlength{\tabcolsep}{0.5em} % for the horizontal padding
	{\renewcommand{\arraystretch}{1.5}% for the vertical padding
		\begin{tabular}{|M{3cm}|M{12cm}|}
			\hline
			\multicolumn{2}{|c|}{\textbf{\textcolor{red}{Informe}}}                                                                                                      \\
			\hline
			\textbf{\textcolor{red}{Latex}} & \textcolor{blue}{El informe está en formato PDF desde Latex,  con un formato limpio (buena presentación) y facil de leer.} \\
			\hline
		\end{tabular}
	}
\end{table}

\subsection{\textcolor{red}{Rúbrica para el contenido del Informe y demostración}}
\begin{itemize}
	\item El alumno debe marcar o dejar en blanco en celdas de la columna \textbf{Checklist} si cumplio con el ítem correspondiente.
	\item Si un alumno supera la fecha de entrega, su calificación será sobre la nota mínima aprobatoria, siempre y cuando cumpla con todos los items.
	\item El alumno debe autocalificarse en la columna \textbf{Estudiante} de acuerdo a la siguiente tabla:

	      \begin{table}[ht]
		      \caption{Niveles de desempeño}
		      \begin{center}
			      \begin{tabular}{ccccc}
				      \hline
				                      & \multicolumn{4}{c}{Nivel}                                                              \\
				      \cline{1-5}
				      \textbf{Puntos} & Insatisfactorio 25\%      & En Proceso 50\% & Satisfactorio 75\% & Sobresaliente 100\% \\
				      \textbf{2.0}    & 0.5                       & 1.0             & 1.5                & 2.0                 \\
				      \textbf{4.0}    & 1.0                       & 2.0             & 3.0                & 4.0                 \\
				      \hline
			      \end{tabular}
		      \end{center}
	      \end{table}

\end{itemize}

\begin{table}[H]
	\caption{Rúbrica para contenido del Informe y demostración}
	\setlength{\tabcolsep}{0.5em} % for the horizontal padding
	{\renewcommand{\arraystretch}{1.5}% for the vertical padding
		%\begin{center}
		\begin{tabular}{|M{2.3cm}|M{5cm}|M{1.2cm}|M{1.5cm}|M{1.8cm}|M{1.4cm}|}
			\hline
			\multicolumn{2}{|c|}{Contenido y demostración} & Puntos                                                                                                                                                                                                        & Checklist & Estudiante & Profesor   \\
			\hline
			\textbf{1. GitHub}                             & Hay enlace URL activo del directorio para el laboratorio hacia su repositorio GitHub con código fuente terminado y fácil de revisar.                                                                          & 2         & X          & 2        & \\
			\hline
			\textbf{2. Commits}                            & Hay capturas de pantalla de los commits más importantes con sus explicaciones detalladas. (El profesor puede preguntar para refrendar calificación).                                                          & 4         & X          & 3        & \\
			\hline
			\textbf{3. Código fuente}                      & Hay porciones de código fuente importantes con numeración y explicaciones detalladas de sus funciones.                                                                                                        & 2         & X          & 2        & \\
			\hline
			\textbf{4. Ejecución}                          & Se incluyen ejecuciones/pruebas del código fuente explicadas gradualmente.                                                                                                                                    & 2         & X          & 1.5      & \\
			\hline
			\textbf{5. Pregunta}                           & Se responde con completitud a la pregunta formulada en la tarea. (El profesor puede preguntar para refrendar calificación).                                                                                   & 2         & X          & 2        & \\
			\hline
			\textbf{6. Fechas}                             & Las fechas de modificación del código fuente estan dentro de los plazos de fecha de entrega establecidos.                                                                                                     & 2         & X          & 2        & \\
			\hline
			\textbf{7. Ortografía}                         & El documento no muestra errores ortográficos.                                                                                                                                                                 & 2         & X          & 1.5      & \\
			\hline
			\textbf{8. Madurez}                            & El Informe muestra de manera general una evolución de la madurez del código fuente,  explicaciones puntuales pero precisas y un acabado impecable. (El profesor puede preguntar para refrendar calificación). & 4         & X          & 4        & \\
			\hline
			\multicolumn{2}{|c|}{\textbf{Total}}           & 20                                                                                                                                                                                                            &           & 18         &            \\
			\hline
		\end{tabular}
		%\end{center}
		%\label{tab:multicol}
	}
\end{table}

\section{Referencias}
\begin{itemize}
	\item Aedo, M. y Castro, E. (2021). FUNDAMENTOS DE PROGRAMACIÓN 2 - Tópicos de Programación Orientada a Objetos. Editorial UNSA.
\end{itemize}

%\pagebreak
%\bibliographystyle{apalike}
%\bibliographystyle{IEEEtranN}
%\bibliography{bibliography}

\end{document}